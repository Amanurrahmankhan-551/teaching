A LTI system has an impulse response
\begin{equation*}
  h(t) = e^{-4(t - 1)} u(t-1).
\end{equation*}
Use two different methods to find the response of the system to
the input
\begin{equation*}
  x(t) = \sin(4 \pi t).
\end{equation*}

{\it Method 1:  for a LTI system the response to a complex
exponential input $x_c(t) = e^{j \omega_0 t}$ is $y_c(t) =
H(\omega_0) e^{j \omega_0 t}$ (you should be able to prove this
using convolution), where $H(\omega)$ is the frequency response of
the system (the Fourier transform of the impulse response $h(t)$).
By writing $x(t) = \sin(4 \pi t)$ in terms of complex exponentials
you can use this to find the output.  Note that it is possible to
massage the result so that the output signal is a real-valued
sinusoid at frequency $\omega_0 = 4 \pi$ (with some scaling and
shift which you can determine).}
\newline

{\it Method 2:  The action of the system can be represented as
$Y(\omega) = H(\omega) X(\omega)$ in the frequency domain.  Take
the Fourier transforms of $h(t)$ and $x(t)$, and multiply them to
get $Y(\omega)$.  This should consist of two Dirac delta functions
with different sizes, one at $\omega = -4 \pi$ and one at $\omega
= 4 \pi$.  You will need to use the sifting property to find the
sizes of these delta functions:  $H(\omega) \delta(\omega -
\omega_0) = H(\omega_0) \delta(\omega - \omega_0)$.  Finding the
inverse Fourier transform of $Y(\omega)$ gives the required
$y(t)$.  Your result should be the same as for the first method.}
