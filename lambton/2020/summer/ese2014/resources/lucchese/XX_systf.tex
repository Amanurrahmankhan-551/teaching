\documentclass[portrait]{seminar}
%\documentclass[a4paper]{article}
\usepackage{graphicx}
\usepackage{amsmath}
\usepackage{amssymb}
%\usepackage{mathtime}
\usepackage{a4wide}
\usepackage{psfrag}

% Default is no solutions.  Compile with solutions with
%   latex "\def\UseOption{sol}\input{file}"
%\usepackage[dummy]{optional}
\usepackage[quest]{optional}

\newcommand{\titem}[1]{\item
  \opt{quest}{\input{q#1} \begin{center} \end{center}}
  \opt{sol}{\vspace*{2ex} \input{a#1} }
  \hrule \vspace*{1ex}
}

\parindent=0ex
\parskip=1ex

\newcommand{\conv}{\ast}
\newcommand{\Nconv}{\bigcirc \hspace{-2.17ex} \mbox{\scriptsize N}
  \hspace{1.0ex}}
\newcommand{\ztpair}{{\stackrel{\cal Z}{\longleftrightarrow}}}

% Seminar options
\slidesmag{1}
\slideframe{none}
\centerslidesfalse

\pagestyle{empty}

\begin{document}
\begin{slide*}

  \begin{center}
    {\LARGE Tutorial:  Systems in time and frequency domains} \newline
    Answers to these questions will {\it not} be made available.
  \end{center} 
  \vspace*{0.5cm}
  The italicised blocks contain hints and guidelines on how to
  approach the questions, and would not appear in an exam.  If you
  understand the concepts required in these questions, then in my
  opinion you understand the content of the course.  Note that it
  might take a few hours to work through these questions
  properly. 
  \vspace*{0.25cm} \hrule \vspace*{0.3cm}
  
  \begin{enumerate}
    \titem{systf1}
    \titem{systf2}
    \titem{systf3}
  \end{enumerate}
  
\end{slide*}
\end{document}
