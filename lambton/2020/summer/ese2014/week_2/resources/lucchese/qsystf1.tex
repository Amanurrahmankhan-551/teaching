A system has an impulse response
\begin{equation*}
  h(t) = e^{-4(t - 1)} u(t-1).
\end{equation*}
Find the response of the system to the input
\begin{equation*}
  x(t) = e^{-2t} u(t)
\end{equation*}
using
\begin{enumerate}
\item Time domain convolution \newline

{\it Straightforward, and you know how to do this:  the output is
$y(t) = h(t) \conv x(t)$, which you find by integration.  I find
it easier to shift $h(t)$ backwards by one so that it starts at
the origin, do the convolution, and shift the answer forwards by
one to get the required answer.  This uses the property that
$y(t+1) = h(t+1) \conv x(t)$.} \newline

\item Frequency domain multiplication. \newline

{\it Convolution in the time domain corresponds to multiplication
in the frequency domain:  $Y(\omega) = H(\omega) X(\omega)$.  Find
Fourier transforms, do the multiplication, and then find the
inverse transform.  You'll probably need a partial fraction
expansion to find this inverse from tables.  Your answer should be
the same as in the previous case.} \newline

\end{enumerate}
